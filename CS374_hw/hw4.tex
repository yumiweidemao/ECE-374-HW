% ---------
%  Compile with "pdflatex hw0".
% --------
%!TEX TS-program = pdflatex
%!TEX encoding = UTF-8 Unicode

\documentclass[11pt]{article}
\usepackage{jeffe,handout,graphicx}
\usepackage[utf8]{inputenc}		% Allow some non-ASCII Unicode in source
\usepackage{tikz}

% =========================================================
%   Define common stuff for solution headers
% =========================================================
\Class{CS/ECE 374}
\Semester{Spring 2023}
\Authors{1}
\AuthorOne{William Cheng}{shihuac2@illinois.edu}
%\Section{}

% =========================================================
\begin{document}

% ---------------------------------------------------------


\HomeworkHeader{4}{1}	% homework number, problem number

\begin{solution}
\begin{enumerate}[(a)]
\item
\begin{enumerate}[i.]
\item
Let $F=\{ 01^k | k > 0 \}$. Let $x,y$ be arbitrary strings in $F$. Then $x=01^i$ and $y=01^j$ for some positive integers $i\neq j$. Let $z=2^i$. Then $xz=01^i2^i\in L$ by definition of $L$. However, $yz=01^j2^i \notin L$ because $1+j\neq 1+i$. Therefore, $F$ is a fooling set for $L$. Since $F$ is infinite, $L$ is not regular.
\item
Let $F$ be the language described by $0^+1$. Let $x,y$ be arbitrary strings in $F$. Then $x=0^i1$ and $y=0^j1$ for some positive integers $i\neq j$. Let $z=0^i$. Then $xz=0^i10^i\notin L$ because the two blocks of $0$'s have equal lengths. But $yz=0^j10^i\in L$ since $i\neq j$. Therefore, $F$ is a fooling set for $L$. Since $F$ is infinite, $L$ is not regular.
\item
Let $f(h)=\lceil h\log_{2}{h}\rceil$. Then
\begin{align*}
f(h+1)-f(h)&=\lceil (h+1)\log_{2}{(h+1)}\rceil - \lceil h\log_{2}{h}\rceil \\
&\geq \lceil (h+1)\log_{2}{(h+1)}\rceil - \lceil h\log_{2}{(h+1)}\rceil\\
&\geq (h+1)\log_{2}{(h+1)} - h\log_{2}{(h+1)} - 1\\
&\geq \log_{2}{(h+1)} - 1\\
&> \log_{2}{(h+1)} - 2\\
&> \log_{2}{(\frac{h+1}{4})}
\end{align*}
Let $n$ be an arbitrary positive integer. Let $h = 4*2^n - 1$. Then $n=\log_{2}{(\frac{h+1}{4})}$. Therefore $f(h+1) - f(h) > n$. Consider the set $F_n=\{0^{f(h)}, 0^{f(h)+1}, ..., 0^{f(h+1)-1}\}$. It has at least $n$ elements.\\
Consider arbitrary strings $x,y\in F_n$, $x=0^{f(h)+i}$ and $y=0^{f(h)+j}$, such that $i,j$ are non-negative integers and $i<j$. Let $z=0^{f(h+1)-f(h)-j}$. Then
\begin{itemize}
\item $yz=0^{f(h+1)}$. $yz\in L$ by definition.
\item $xz=0^{f(h+1)+i-j}$. $f(h+1)+i-j > f(h)$ since $j<f(h+1)-f(h)$ by definition of $F_n$. $f(h+1)+i-j < f(h+1)$ since $i<j$. Therefore $xz\notin L$ because the number of zeros is not a possible value of $f$.
\end{itemize}
Therefore, $F_n$ is a fooling set of length $\geq n$ for $L$. Since $F_n$ exists for any $n>0$, $L$ is not regular.
\end{enumerate}
\item
Let $F_k=\{ w\in\{ 0,1 \}^* \text{ } | \text{ } \lvert w \rvert = k \}$. Let $x, y$ be arbitrary strings in $F_k$. To find the distinguishing suffix $z$, let $f$ be a function such that $z=f(x, y)$. $f$ is defined recursively as follows:
\begin{itemize}
\item If $\# (0, x) \neq \# (0, y)$, then $z=0^{\#(1, x)}1^{\#(0, x)}$.
\item If $\# (0, x) = \# (0, y)$, and $x=ax', y=ay'$ where $a\in \{0,1\}$, then $z=(f(x', y')\cdot 01)$.
\end{itemize}
\emph{Explanation.} $F_k$ is a set containing all bitstrings of length $k$. Intuitively, the function finds from left to right the first position where $x$ and $y$'s suffixes, $x'$ and $y'$, have different number of $0$'s (thus different number of $1$'s since their length is identical). Then, a suffix $z'$ is added to $x'$ such that $x'z'$ has equal number of $0$'s and $1$'s. It is easily known that $y'z'$ does not have equal number of $0$'s and $1$'s because the number of $0$'s and $1$'s in $y'$ is different from $x'$. Finally, a block of $01$'s is added to $z'$ to make $z$, until $\lvert x'z \rvert = \lvert y'z \rvert = 2k$. The addition of the $01$ block will still make $x'z$ have equal number of $0$'s and $1$'s, and make $y'z$ have unequal number of $0$'s and $1$'s. Therefore, by definition, $xz\notin L_k$ and $yz\in L_k$. Thus, $F_k$ is a fooling set for $L_k$. $\lvert F_k \rvert = 2^k$.
\item
$L'$ is regular because it is finite and can be represented with the regular expression $\sum{w\in L'}$. Then $L'\cap L$ is also regular since it is a subset of $L'$ and thus finite.
Assume $L\setminus L'$ is regular.
Then $(L\setminus L')\cup (L\cap L')=L$ must be regular by closure of union operation in regular languages.
This is a contradiction because $L$ is not regular. Therefore, $L\setminus L'$ must be non-regular.\\

\emph{Example: }Let $L=\{0^k1^k|k\geq 0\}$. Let $L'=L(0^*1^*)$. Then $L$ is non-regular and $L'$ is regular. $L\setminus L'=\emptyset$ which is a regular language.
\end{enumerate}
\end{solution}

% ---------------------------------------------------------
\HomeworkHeader{4}{2}

\begin{solution}
\begin{enumerate}[(a)]
\item
\begin{equation*}
S\to \epsilon \mid aSc \mid aSd \mid bSc \mid bSd
\end{equation*}
\emph{Explanation. }$\epsilon$ is clearly in this language. For any $S$ in this language, we remove one letter from $\{a, b\}$ and one letter from $\{c, d\}$, and the resulting string is still in this language since $i+j=k+l\implies i+j-1=k+l-1$. Therefore, we can define $S$ recursively as shown above.
\item
\begin{equation*}
S\to \epsilon \mid 0S222 \mid 1S222
\end{equation*}
\emph{Explanation. }$\epsilon$ is clearly in this language. For any $S$ in this language, we remove three $2$'s at the end, and remove one $0$ or $1$ in the front, the resulting string is still in the language since $3(i+j)-3=3(i+j-1)$. Therefore, we can define $S$ recursively as shown above.
\item
\begin{align*}
&S\to P \mid B\#S \mid S\#B \mid R\\
&B\to \epsilon \mid 1B \mid 0B &&\text{bitstrings}\\
&P\to \epsilon \mid 0 \mid 1 \mid 0P0 \mid 1P1 &&\text{palindromes}\\
&R\to \#C\# \mid 0R0 \mid 1R1 &&\text{strings in $L$ with $x_1=x_k^R$}\\
&C\to B \mid C\#B &&\text{bitstrings separated by $\#$'s}
\end{align*}
\emph{Explanation. }Consider the two cases for a string $x_1\#x_2\#...\#x_k$ to be in $L$.
\begin{enumerate}[i.]
\item There exists some $i=j$, such that $x_i = x_j^R$. In other words, $x_i$ is a palindrome. From the start symbol $S$, all these strings can be described with $S\to P \mid B\#S \mid S\#B$, where $P$ represents palindromes and $B$ represents bitstrings.
\item There exists some $i\neq j$, such that $x_i = x_j^R$.\\
First consider the subcase where $x_1$ and $x_k$, the first and last substrings separated by $\#$, are reversals of each other. In other words, $(i,j)=(1,k)$. All these strings are described by the non-terminal $R$.\\
Then, all strings in case (ii) can be described by $S\to R \mid B\#S \mid S\#B$.
\end{enumerate}
Then consider an arbitrary string in $L$ with $x_i=x_j^R$ for some $i, j$. Either $i=j$ or $i\neq j$, so it would fall into exactly one case above. Therefore, the union of these cases describes $L$, which can be represented using the CFG $S\to P \mid B\#S \mid S\#B \mid R$.
\item
\begin{align*}
&S\to M \mid L &&\{ 1^m0^n \mid m\neq n \}\\
&M\to 1M \mid 1E &&\{ 1^m0^n \mid m>n \}\\
&L\to L0 \mid E0 &&\{ 1^m0^n \mid m<n \}\\
&E\to \epsilon \mid 1E0 &&\{ 1^n0^n \mid n \geq 0 \}
\end{align*}
\emph{Explanation. }The complement of $L$ can be described as $\{1^m0^n \mid m\neq n\}$. Then consider the cases where $m$ is $M$ore than $n$ and $m$ is $L$ess than $n$. The case where $m$ is $E$qual to $n$ is used as a basis. The non-terminal $L$ adds at least one $0$'s to the basis $E$ so that $m<n$ always holds. The non-terminal $M$ adds at least one $1$'s to the basis $E$ so that $m>n$ always holds. Finally, $S$ is produced from the union of $M$ and $L$. The union of $m<n$ and $m>n$ produces $m\neq n$.
\end{enumerate}
\end{solution}




\end{document}
