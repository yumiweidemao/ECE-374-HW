% ---------
%  Compile with "pdflatex hw0".
% --------
%!TEX TS-program = pdflatex
%!TEX encoding = UTF-8 Unicode

\documentclass[11pt]{article}
\usepackage{jeffe,handout,graphicx}
\usepackage[utf8]{inputenc}		% Allow some non-ASCII Unicode in source

% =========================================================
%   Define common stuff for solution headers
% =========================================================
\Class{CS/ECE 374}
\Semester{Spring 2023}
\Authors{1}
\AuthorOne{William Cheng}{shihuac2@illinois.edu}
%\Section{}

% =========================================================
\begin{document}

% ---------------------------------------------------------


\HomeworkHeader{1}{1}	% homework number, problem number

\begin{solution} 
\begin{enumerate}[(a)]
\item 
\textbf{Claim:} \emph{For $a\geq max(\frac{\gamma}{1-(c_{1}^{2}+c_{2}^{2}+c_{3}^{2})}, 1)$}\footnote{For the induction step, the condition $a \geq \frac{\gamma}{1-(c_{1}^{2}+c_{2}^{2}+c_{3}^{2})}$ is sufficient, but $a\geq 1$ is necessary for the base case where $n=1$.} and 
$b=0$, and for all $n\geq1$, \\$T(n)\leq an^{2}+b$.\\\\
\emph{Proof.} \textbf{Base case: } For $1\leq n\leq\frac{1}{c_{1}}$, $T(n)=1\leq an^{2}+b$ for $a\geq 1$ by definition.\\
\textbf{Inductive hypothesis: } Let $n > \frac{1}{c_{1}}$. Assume $T(k)\leq ak^{2}+b$ for all $1\leq k < n$.\\
\textbf{Inductive step: }
\begin{align*}
T(n)	&= T(\lfloor c_{1}n\rfloor) + T(\lfloor c_{2}n\rfloor) + T(\lfloor c_{3}n\rfloor) + \gamma n^{2}\\
	&\leq a(\lfloor c_{1}n\rfloor)^{2} + a(\lfloor c_{2}n\rfloor)^{2} + a(\lfloor c_{3}n\rfloor)^{2} + 3b + \gamma n^{2}  
		&& \text{by induction}\\
	&\leq a(c_{1}n)^{2} + a(c_{2}n)^{2} + a(c_{3}n)^{2} + 3b + \gamma n^{2}  
		&& \text{by definition of floor operation}\\
	&\leq ((c_{1}^{2}+c_{2}^{2}+c_{3}^{2})a + \gamma)n^{2} + 3b \leq an^{2} + b\\
\text{provided that} \\
	&((c_{1}^{2}+c_{2}^{2}+c_{3}^{2})a + \gamma) \leq a \iff 
		a\geq \frac{\gamma}{1-(c_{1}^{2}+c_{2}^{2}+c_{3}^{2})} \\
	& 3b \leq b \iff b = 0 && \text{since } b \geq 0 \\
\end{align*}\
Hence, $T(n) \leq an^{2} + b$ for any $a\geq max(\frac{\gamma}{1-(c_{1}^{2}+c_{2}^{2}+c_{3}^{2})}, 1)$ and 
$b=0$ for all $n\geq 1$. Thus, $T(n) = O(n^{2})$ for all $n \geq 1$.

\item The asymptotic upper bound is determined by the rightmost leaf node of the recursion tree. The value of the node is $c_{3}^{k}n$ where $k$ is the depth. $c_{3}^{k}n = 1$ since it's the leaf node, which gives $k = \log_{\frac{1}{c_{3}}}{n}$. Hence, the upper bound of the tree depth is $\log_{\frac{1}{c_{3}}}{n}$.

\item 
$a\geq max(\frac{\gamma}{1-\sum_{i=1}^{k}{c_{i}^{2}}}, 1)$ \\
The upper bound of the depth of the recursion tree is $\log_{\frac{1}{c_{k}}}{n}$

\end{enumerate}
\end{solution}

% ---------------------------------------------------------
\HomeworkHeader{1}{2}

\begin{solution}
\begin{enumerate}[(a)]
\item
\textbf{Claim: }\emph{For any $w\in L_{1}$ with $n = \lvert w \rvert \geq 0$, $w\in L_{ee}$}.\\
\emph{Proof.} \textbf{Base case: } For $n = \lvert w \rvert = 0$, $w=\epsilon$. $\#(0, \epsilon)=0$ and $\#(1, \epsilon)=0$. Since $0$ is an even number, $w\in L_{ee}$.\\
\textbf{Inductive hypothesis: } Let $n > 0$. Assume that all strings $x\in L_{1}$ with $0 \leq \lvert x \rvert < n$ are in $L_{ee}$.\\
\textbf{Inductive step: } Let $w$ be a string of length $n$ in $L_{1}$. By the last property of $L_{1}$, $w$ can only be generated from a string $z\in L_{1}$. Consider the case where $w$ is generated by inserting a $00$ or $11$ into some string $z\in L_{1}$. Then $\lvert z \rvert = n-2$, which implies $z\in L_{ee}$ by induction. Then $z$ has even number of $0$'s and even number of $1$'s by definition of $L_{ee}$. Adding exactly two ones or zeros would still make an even number of $0$'s and $1$'s, therefore $w\in L_{2}$ by definition. Consider the other case where $w$ is generated by concatenating some string $x\in L_{1}$ with $0101$ or $1010$. Then $\lvert x \rvert=n-4$, which implies $x\in L_{ee}$. Then $\#(0, x)$ and $\#(1, x)$ are even. $\#(0, w)=\#(0, x)+2$ and $\#(1, w)=\#(1, x)+2$ are also even, therefore $w\in L_{ee}$.
Since $w\in L_{1}$ implies $w\in L_{ee}$, $L_{1}\subseteq L_{ee}$.

\item
\textbf{Claim: }\emph{For any $w\in L_{ee}$ with $n = \lvert w \rvert \geq 0$, $w\in L_{1}$}.\\
\emph{Proof.} \textbf{Base case: } For $n = \lvert w \rvert = 0$, $w=\epsilon$, $\#(0,w)=\#(1,w)=0$. Since $0$ is an even number, $w\in L_{ee}$. By definition, $\epsilon \in L_{1}$, therefore $w \in L_{1}$.\\
\textbf{Inductive hypothesis: } Let $n > 0$. Assume that all strings $x\in L_{ee}$ with $0 \leq \lvert x \rvert < n$ are in $L_{1}$.\\
\textbf{Inductive step: } Let $w$ be a string of length $n$ in $L_{ee}$. By definition, the number of $1$'s and the number of $0$'s in $w$ are even. Consider the case where $w$ contains at least two consecutive $1$'s or $0$'s. Then $w$ can be written as $x11y$ or $x00y$. $xy\in L_{ee}$ because taking exactly $2$ ones or zeros out of $w$ would still make an even number of ones and zeros. Then $xy\in L_{1}$ by induction. By definition of $L_{1}$, $xy\in L_{1}$ implies $x00y\in L_{1}$ and $x11y\in L_{1}$, thus $w\in L_{1}$. Then consider the case where $w$ does not contain two consecutive $1$'s or $0$'s. In this case $w$ must be alternating $0$'s and $1$'s, and $\lvert w \rvert \geq 4$ to ensure even number of $0$'s and $1$'s. Thus, $w$ is in the form $z0101$ or $z1010$ where $\lvert z \rvert = \lvert w \rvert - 4$. Since $\#(0, z)=\#(0, w)-2$ and $\#(1, z)=\#(1, w)-2$, $z$ must have even number of $1$'s and even number of $0$'s, therefore $z\in L_{ee}$, which implies $z\in L_{1}$ by induction. By definition of $L_{1}$, $z\in L_{1}$ implies $z0101\in L_{1}$ and $z1010\in L_{1}$, therefore $w\in L_{1}$. Since $w\in L_{ee}$ implies $w\in L_{1}$ in all cases, $L_{ee}\subseteq L_{1}$.

\item
\begin{itemize}
\item A string in $L_{eo} - L_{2}$: $010$
\item $L'=010(1010)^{*}$.
\end{itemize}
\emph{Proof.} \textbf{$L'\subseteq L_{eo}$: }For any $w\in L'$, $\#(0, w)=2+2n$ which is even and $\#(1, w)=1+2n$ which is odd, where $n$ is the number of $1010$'s after $010$ in $w$, therefore $w\in L_{eo}$. \textbf{$L'\cap L_{2}=\emptyset$: }Let $w$ be an arbitrary string in $L'$. Assume $w\in L_{2}$. Then $010$ must be in $L_{2}$ since $w$ can only be obtained by concatenating $010$ with some number of $1010$'s by definition of $L_{2}$. However, $010$ cannot be in $L_{2}$ since it is not in the form $x00y$ or $x11y$ or $x1010$ or $x0101$. Hence, $w\in L'$ implies $w\notin L_{2}$.
\end{enumerate}
\end{solution}




\end{document}
