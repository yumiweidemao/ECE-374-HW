% ---------
%  Compile with "pdflatex hw0".
% --------
%!TEX TS-program = pdflatex
%!TEX encoding = UTF-8 Unicode

\documentclass[11pt]{article}
\usepackage{jeffe,handout,graphicx}
\usepackage[utf8]{inputenc}		% Allow some non-ASCII Unicode in source

% =========================================================
%   Define common stuff for solution headers
% =========================================================
\Class{CS/ECE 374}
\Semester{Spring 2023}
\Authors{1}
\AuthorOne{William Cheng}{shihuac2@illinois.edu}
%\Section{}

% =========================================================
\begin{document}

% ---------------------------------------------------------


\HomeworkHeader{11}{1}	% homework number, problem number

\begin{solution}
\begin{enumerate}[(a)]
\item Suppose $L_{\text{regular}}$ is decidable and there is an algorithm $\textsc{DecideLRegular}$ that correctly decides the language $L_{\text{regular}}$. Then we can solve the halting problem as follows:
\begin{algo}
	\textul{\textsc{DecideHalt($\langle M,w \rangle$):}}\+
\\	Encode a Turing machine $M'$:\+
\\	\textul{$M'(x):$}\+
\\	run $M$ on input $w$
\\	return \textsc{True}\-\-
\\	if $\textsc{DecideLRegular($\langle M' \rangle$)}$\+
\\	return \textsc{True}\-
\\	return \textsc{False}\-
\end{algo}
\begin{itemize}
\item Suppose $M$ halts on input $w$. Then $M'$ accepts every input string $x$, including any regular string. Then $\textsc{DecideLRegular}$ accepts the encoding $\langle M' \rangle$. So $\textsc{DecideHalt}$ correctly accepts the encoding $\langle M,w \rangle$.
\item Suppose $M$ doesn't halt on input $w$. Then $M'$ diverges on every input string $x$, including any regular string. Then $\textsc{DecideLRegular}$ rejects the encoding $\langle M' \rangle$. So $\textsc{DecideHalt}$ correctly rejects the encoding $\langle M,w \rangle$.
\end{itemize}
However, $\textsc{DecideHalt}$ cannot be correct because $\textsc{Halt}$ is undecidable. Therefore, the algorithm $\textsc{DecideLRegular}$ does not exist, and $L_{\text{regular}}$ is undecidable.

\item Suppose we have an algorithm $\textsc{DecideHalt}(\langle M \rangle)$ that correctly decides if $M$ halts on blank input. Then we can decide $L_u$ as follows:
\begin{algo}
	\textul{\textsc{DecideLU}($\langle M,w \rangle$):}\+
\\	Encode a Turing machine $M'$:\+
\\	$M'$ writes $w$ on the tape and simulates $M$\-
\\	if $\textsc{DecideHalt}(\langle M' \rangle)$\+
\\	run $w$ on $M$
\\	if $M$ accepts $w$\+
\\	return \textsc{True}\-\-
\\	return \textsc{False}\-
\end{algo}
\begin{itemize}
\item Suppose $M$ accepts $w$. Then $M'$ will halt and $\textsc{DecideHalt}$ accepts $\langle M' \rangle$. Then we run $w$ on $M$ and since $M$ accepts $w$, $\textsc{DecideLU}$ correctly accepts $\langle M,w \rangle$.
\item Suppose $M$ does not accept $w$. Then there are two cases: 1) $M$ rejects $w$. Then $M'$ halts and \textsc{DecideHalt} accepts  $\langle M' \rangle$. Then we run $w$ on $M$ and since $M$ rejects $w$, \textsc{DecideLU} correctly rejects $\langle M,w \rangle$. 2) $M$ diverges on $w$. Then \textsc{DecideHalt} rejects $\langle M' \rangle$, and \textsc{DecideLU} correctly rejects $\langle M,w \rangle$.
\end{itemize}
Therefore, the reduction from $L_u$ to $L_{\text{HALT}}$ is correct.

\item We begin by showing that $L_{\text{nonemptylang}}=\Sigma^* \setminus L_{\text{emptylang}}$ is undecidable. Then, we prove that $L_{\text{nonemptylang}}$ is recursively enumerable. Finally, we use Lemma 5 in Jeff's notes 7.3 to prove that $L_{\text{emptylang}}$ is not recursively enumerable by contradiction.
\begin{itemize}
\item Let $L_{\text{nonemptylang}}=\Sigma^* \setminus L_{\text{emptylang}}$. Suppose there is an algorithm $\textsc{DecideNonEmptyLang}(\langle M \rangle)$ that decides if $M$ accepts at least one string. We can reduce the Halting problem to \textsc{DecideNonEmptyLang} as follows:
\begin{algo}
	\textsc{\textul{DecideHalt($\langle M,w \rangle$):}}\+
\\	Encode a Turing machine $M'$:\+
\\	\textul{$M'(x):$}\+
\\	run $M$ on input $w$
\\	return \textsc{True}\-\-
\\	if $\textsc{DecideNonEmptyLang($\langle M' \rangle$)}$\+
\\	return \textsc{True}\-
\\	return \textsc{False}\-
\end{algo}
If $M$ halts on $w$, then $M'$ accepts every input and clearly accepts at least one string, so $\textsc{DecideNonEmptyLang}$ accepts the encoding $\langle M' \rangle$ and $\textsc{DecideHalt}$ correctly accepts the encoding $\langle M,w \rangle$. If $M$ does not halt on $w$, then $M'$ diverges on every input and $L(M')=\emptyset$, so $\textsc{DecideNonEmptyLang}$ rejects the encoding $\langle M' \rangle$ and $\textsc{DecideHalt}$ correctly rejects the encoding $\langle M,w \rangle$. We know that the Halting problem is undecidable, therefore $L_{\text{nonemptylang}}$ is undecidable.

\item Let $M'(\langle M \rangle)$ be a Turing machine that operates as follows: $M'$ enumerates over $\Sigma^*$, and for each string $w\in \Sigma^*$ checks if $M$ accepts $w$. $M'$ accepts $\langle M \rangle$ if $M$ accepts $w$. It is obvious that $M'$ will accept $\langle M \rangle$ if $M$ accepts at least one string, and will diverge on $\langle M \rangle$ if $M$ does not accept any string, and therefore $M'$ accepts $L_{\text{nonemptylang}}$. So $L_{\text{nonemptylang}}$ is recursively enumerable.

\item Lemma 5 in Jeff's notes 7.3: an acceptable (recursively enumerable) language is decidable if and only if its complement is acceptable. Assume $L_{\text{emptylang}}$ is recursively enumerable. Then $L_{\text{nonemptylang}}$ is decidable by Lemma. However, we have proved that $L_{\text{nonemptylang}}$ is undecidable, which is a contradiction.
\end{itemize}
Therefore, $L_{\text{emptylang}}$ is not recursively enumerable.
\end{enumerate}
\end{solution}

% ---------------------------------------------------------
\HomeworkHeader{11}{2}

\begin{solution}
\begin{enumerate}[(a)]
\item Let $x_e$ be a Boolean variable for $e\in E$. If we add $e$ to the answer set $M$ then $x_e=1$. In a perfect matching, each vertex is incident to exactly one edge. We use the following functions to generate Boolean expressions that checks if a vertex $v$ has at least one incident edge/at most one incident edge.
\begin{algo}
	\textsc{\textul{AtLeastOne($v$):}}\+
\\	find incident edges $e_1, ..., e_k$ of $v$
\\	if there is no incident edge return \textsc{False}
\\	return $x_{e_1}\lor ... \lor x_{e_k}$\-
\\
\\	\textsc{\textul{AtMostOne($v$):}}\+
\\	find incident edges $e_1, ..., e_k$ of $v$
\\	if there is $0$ or $1$ incident edge return \textsc{True}
\\	return $(\lnot x_{e_1}\lor \lnot x_{e_2})\land...\land(\lnot x_{e_1}\lor \lnot x_{e_k})\land(\lnot x_{e_2}\lor \lnot x_{e_3})\land...\land(\lnot x_{e_{k-1}}\lor \lnot x_{e_k})$
\end{algo}
The Boolean expression in \textsc{AtLeastOne} evaluates to $0$ if no incident edge of $v$ is added to $M$, otherwise $1$. The Boolean expression in \textsc{AtMostOne} enumerates all pairs of incident edges and checks if at least one pair of edges is in $M$. If there is at least a pair, it evaluates to $0$. Otherwise, it evaluates to $1$. Following is the algorithm to check if a perfect matching exists using an algorithm for SAT.
\begin{algo}
	\textsc{\textul{ExactlyOne($v$):}}\+
\\	return $\textsc{AtLeastOne}(v) \land \textsc{AtMostOne}(v)$\-
\\
\\	\textsc{\textul{DecidePerfectMatching($G$):}}\+
\\	let $v_1, v_2, ..., v_n$ be the vertices of $G$
\\	$\phi\gets \textsc{ExactlyOne}(v_1) \land ... \land \textsc{ExactlyOne}(v_n)$
\\	return $\textsc{DecideSAT}(\phi)$
\end{algo}
\emph{Correctness:} If there exists a set of variables $x_{e_1},...,x_{e_m}$ that satisfies the Boolean expression $\phi$, then every vertex in $G$ must have exactly one incident edge: $\emph{deg}(v)=1$ for all $v\in V$ since it's the definition of $\phi$. Let $M=\{e_i\mid e_i\in E, x_{e_i}=1\}$. Then $M$ is a matching of $G$ because no two edges share a vertex (otherwise there exists $v_i$ where $\emph{deg}(v_i)>1$). $|M|=\frac{\sum_{v\in V}{\emph{deg}(V)}}{2}=\frac{|V|}{2}$, therefore $M$ is a perfect matching.

If there exists a perfect matching $M$ of $G$, then every vertex in $G$ must have exactly one incident edge. We make $x_{e_i}=1$ for all $e_i\in M$ and $x_{e_i}=0$ otherwise. Then the set of variables $x_{e_1},...,x_{e_m}$ satisfies $\phi$ since $\textsc{ExactlyOne}(v)=1$ for all $v\in V$ by definition.

Since an instance of PerfectMatching is a "yes" instance if and only if the reduction $\phi$ is a "yes" instance of SAT, the reduction is correct.

\emph{Time complexity:} \textsc{AtLeastOne} takes $O(|E|)$ time and \textsc{AtMostOne} takes $O(|E|^2)$ time. We compute \textsc{ExactlyOne} for $|V|$ times. Therefore, computing a formula for $\phi$ takes $O(|V||E|^2)$ time, which is polynomial.

This does not prove that PerfectMatching is NP-complete because this shows $\text{PerfectMatching} \le_{P} \text{SAT}$. If we want to prove PerfectMatching is NP-complete we need to reduce SAT to PerfectMatching, which is $\text{SAT} \le_{P} \text{PerfectMatching}$.

\item We begin by proving that EIGHT is in NP, and then reduce the Hamiltonian cycle problem to EIGHT to prove EIGHT is NP-hard:
\begin{itemize}
\item We use a subgraph as a certificate and check if the subgraph is an eight-graph. We can use the following certifier: check if there are $2k-1$ nodes in the subgraph, run DFS on an arbitrary node and get a DFS tree of the subgraph, and look for back-edges in the DFS tree to identify cycles. If there are $2k-1$ nodes in the subgraph and exactly two cycles of size $k$ and exactly one node that is in both cycles, then return true. The certifier runs in polynomial time, therefore EIGHT is in NP.

\item We can reduce Hamiltonian cycle to EIGHT by constructing a graph $G'$: first, we add two disjoint copies of $G$ to $G'$; second, we add a new vertex $a$ to $G'$ and connect $a$ to all the other nodes in $G'$. Let $k=|V|+1$.
\begin{itemize}
\item If $G$ has a Hamiltonian cycle, then $G'$ has an eight-graph on $2k-1$ nodes. Let the Hamiltonian cycles of the two copies of $G$ be $C_1$ and $C_2$. Let $u_1, v_1$ and $u_2, v_2$ be two connected nodes in $C_1$ and $C_2$. Then the graph $C_1+C_2-u_1v_1-u_2v_2+u_1a+av_1+u_2a+av_2$ is an eight-graph on $2k-1$ nodes: it contains nodes from $C_1$, $C_2$ and $a$, so its size is $2|V|+1=2k-1$. It contains two cycles of size $|V|+1=k$: each cycle is obtained by removing an edge in the Hamiltonian cycle and adding $a$ to the cycle. The vertex $a$ is the only shared vertex of the two cycles since the two Hamiltonian cycles are originally disjoint.
\item If $G'$ has an eight-graph on $2k-1$ nodes, then $G$ has a Hamiltonian cycle. $a$ must be the shared vertex of the two cycles because the two cycles were originally disjoint. Denote the two copies of $G$ as $G_1$ and $G_2$. Let $u_1, v_1$ be the neighbors of $a$ in $G_1$. Let the cycle containing vertices in $G_1$ and $a$ be $C_1$. By definition, $|C_1|=k$. Now we remove $a$ and its incident edges in $C_1$ and add the edge $u_1v_1$ to it to form a new cycle $C_1'$. Then the size of $C_1'$ is $k-1=|V|$. Since $C_1'$ is a cycle whose vertices all belong to $G_1$ and $|C_1'|=|V|$, we conclude that $C_1'$ is a Hamiltonian cycle in $G_1$. Since $G_1$ is a copy of $G$, $G$ must have a Hamiltonian cycle.
\end{itemize}
Therefore, the reduction from Hamiltonian cycle to EIGHT is correct. Since Hamiltonian cycle is NP-hard, EIGHT must also be NP-hard.
\end{itemize}
Since EIGHT is in NP and EIGHT is NP-hard, it can be concluded that EIGHT is NP-complete.
\end{enumerate}
\end{solution}




\end{document}
