% ---------
%  Compile with "pdflatex hw0".
% --------
%!TEX TS-program = pdflatex
%!TEX encoding = UTF-8 Unicode

\documentclass[11pt]{article}
\usepackage{jeffe,handout,graphicx}
\usepackage[utf8]{inputenc}		% Allow some non-ASCII Unicode in source

% =========================================================
%   Define common stuff for solution headers
% =========================================================
\Class{CS/ECE 374}
\Semester{Spring 2023}
\Authors{1}
\AuthorOne{William Cheng}{shihuac2@illinois.edu}
%\Section{}

% =========================================================
\begin{document}

% ---------------------------------------------------------


\HomeworkHeader{9}{1}	% homework number, problem number

\begin{solution}
\begin{enumerate}[(a)]
\item We use a layered construction to keep track of whether Yulie has bought empanadas and whether a gas station has been reached. For the layer where a gas station has not been reached, we need to know which vertices can be reached and which cannot. Therefore, we need to run Dijkstra's algorithm once on $G$ not taking gas stations into account. Following is the algorithm.
\begin{enumerate}[(1)]
\item Run Dijkstra's algorithm on $G$ starting at $s$ to compute $\emph{pred}(v)$, the predecessor of $v$ in the shortest path from $s$ to $v$. Let $\emph{remainingMiles}(v)$ be the maximum number of miles that Yulie's car can run when she reaches $v$. It can be defined as follows:
\begin{equation*}
\emph{remainingMiles}(v)=
\begin{cases}
	D &v=s
\\	\emph{remainingMiles}(p) - l(pv) &\text{otherwise, }p=\emph{pred}(v)
\end{cases}
\end{equation*}
The time complexity of computing $\emph{remainingMiles}$ is $O(1)$ and we can put its computation into each iteration of Dijkstra's algorithm. Since Dijkstra's algorithm explores the shortest path, $\emph{remainingMiles}(v)$ must be the maximum remaining number of miles at $v$.
\item Construct a graph $G'=(V', E')$, and define edge lengths $l'(e)$ for $E'$:
\begin{itemize}
\item $V'=V\times \{NE,E\} \times \{NG,G\}$, where $NE$ means Yulie has not bought empanadas yet, $E$ means Yulie has bought empanadas, $NG$ means a gas station has not been reached, $G$ means a gas station has been reached.
\item For each edge $uv\in E$, we add the following to $E'$ and $l'(e)$. We use $a$ and $b$ to generalize empanadas and gas stations.
\begin{itemize}
\item $(u, a, NG)(v, a, NG)$ if $\emph{remainingMiles}(v)\ge 0$ and $u=\emph{pred}(v)$; $l'(e)=l(uv)$
\item $(u, a, G)(v, a, G)$; $l'(e)=l(uv)$
\item $(u, a, NG)(u, a, G)$ if $u\in Y$; $l'(e)=0$
\item $(u, NE, b)(u, E, b)$ if $u\in X$; $l'(e)=0$
\end{itemize}
\end{itemize}
\item Let $s'=(s,NE,NG)$. Run Dijkstra's algorithm on $G'$ starting at $s'$ using $l'(e)$ as edge weights. Let $\emph{dist}(v)$ be the shortest distance from $s'$ to $v$ in $G'$. Return the minimum of $\emph{dist}((t,E,NG))$ and $\emph{dist}((t,E,G))$.
\end{enumerate}
It is easy to also return a shortest path since Dijkstra's algorithm can keep track of the path. Note that $\emph{dist}(v)=\infty$ if there is not a path from $s$ to $v$ in Dijkstra's algorithm. Therefore the algorithm will return $\infty$ if there is no way to reach the destination. In the new graph $G'$, $|V'|=4|V|$ and $|E'|\le |X|+|Y|+2|E| \le |V|+2|E|$. Since the running time for Dijkstra's algorithm is $O(|E|+|V|\log{|V|})$, the running time of this algorithm is $O(|V|+2|E|+4|V|\log{4|V|})=O(m+n\log{n})$.

%1b
\item 
Let $k=|Y|$. We want to know the "reach" of each gas station to account for the fact that Yulie can only run $R$ miles after reaching a gas station. Therefore, we run Dijkstra's algorithm $k$ times using each gas station as the start node. The layers are concatenated and form a new graph. Then the problem is reduced to an SSSP problem in the new graph. The algorithm is as follows:
\begin{enumerate}[(1)]
\item Run Dijkstra's algorithm $k+1$ times on $G$ starting from $s, y_1, y_2, ..., y_k$ where $y_i\in Y$. Let $\emph{pred}(y_i, v)$ be the predecessor of $v$ in the shortest path from $y_i$ to $v$. Each time Dijkstra's algorithm is run starting from $y_i$, we compute an array $\emph{remainingMiles}(y_i, v)$ using the same technique as in (a) which represents the remaining miles when Yulie reaches $v$ from $y_i$. It is defined as follows:
\begin{equation*}
\emph{remainingMiles}(y_i, v)=
\begin{cases}
	R &v=y_i
\\	\emph{remainingMiles}(y_i, p) - l(pv) &\text{otherwise, }p=\emph{pred}(y_i, v)
\end{cases}
\end{equation*}
Let $G_1, G_2, ..., G_k$ be the resulting DAGs representing the shortest paths starting from $y_1, y_2, ..., y_k$. Then for each $G_i$, we remove all edges $\{uv\mid \emph{remainingMiles}(y_i, v)<0, u=\emph{pred}(y_i, v)\}$.
\item Construct a new graph $G'=(V', E')$:
\begin{itemize}
\item $V'=V\times \{N, R_1, R_2, ..., R_k\} \times \{E, NE\}$. $E$ means Yulie has bought empanadas and $NE$ means Yulie has not bought empanadas. $N$ means Yulie has not reached any gas station yet, $R_i$ means the last gas station reached is $y_i$.
\item For each edge $uv\in E$, we add the following to $E'$ and $l'(e)$. We use $r$ to generalize $\{N, R_1, R_2, ..., R_k\}$ and $a$ to generalize $\{E, NE\}$.
\begin{itemize}
\item $(u, N, a)(v, N, a)$ if $\emph{remainingMiles}(s,v)\ge 0$ and $u=\emph{pred}(s, v)$; $l'(e)=l(uv)$
\item $(u, R_i, a)(v, R_i, a)$ if $uv$ is an edge in $G_i$; $l'(e)=l(uv)$
\item $(u, r, NE)(u, r, E)$ if $u\in X$; $l'(e)=0$
\item $(u, r, a)(u, R_i, a)$ if $u=y_i$; $l'(e)=0$
\end{itemize}
\end{itemize}
\item Let $s'=(s, N, NE)$ and run Dijkstra's algorithm on $G'$ starting from $s'$ using $l'(e)$ as edge weights. Return the minimum of the lengths of the shortest paths from $s'$ to $(t,r,E)$ where $r\in\{N, R_1, R_2, ..., R_k\}$.
\end{enumerate}
The algorithm will return $\infty$ if there is no valid path due to the implementation of Dijkstra's algorithm. The size of the new graph $G'$ is $|V'|=2(k+1)n$ and $|E'|\le (k+1)m + n$. We run Dijkstra $k+1$ times on the original graph and once on $G'$, so the overall time complexity is $O(km+kn\log{n})$ where $k=|Y|$.
\end{enumerate}
\end{solution}

% ---------------------------------------------------------
\HomeworkHeader{9}{2}

\begin{solution}
\begin{enumerate}[(a)]
\item We begin by computing the meta-graph $G^{\text{SCC}}$ of $G$. The weight of each node $w'(u)$ in $G^{\text{SCC}}$ is the sum of all $w(v)$ where $v\in u$ because the eggs at each node can only be collected once. We want to use the algorithm to find the longest path in a DAG (which is mentioned in lecture). In order to convert vertex weights to edge weights, we divide each vertex $v\in G^{\text{SCC}}$ into two halves $v_{in}$ and $v_{out}$ with $v_{in}$ connected to the in-edge and $v_{out}$ connected to the out-edge. The weights of these two edges are $0$. We then add an edge $v_{in}v_{out}$ with weight $l(e)=w'(v)$. Following is the algorithm.
\begin{enumerate}[(1)]
\item Compute meta-graph $G^{\text{SCC}}=(V^{\text{SCC}}, E^{\text{SCC}})$ using DFS. Let $w'(u)$ be the weight of vertex $u$ in $V^{\text{SCC}}$. For each $u\in V^{\text{SCC}}$, compute $w'(u)=\sum_{v\in u}{w(v)}$.
\item Construct graph $G'=(V', E')$ defined as follows:
\begin{itemize}
\item $V'= V^{\text{SCC}} \times \{\emph{in}, \emph{out}\}$
\item $E'=\{(u, \emph{out})(v, \emph{in}) \mid uv\in E^{\text{SCC}}\} \cup \{(u, \emph{in})(u, \emph{out}) \mid u\in V^{\text{SCC}}\}$
\item For all edges in the form $e=(u, \emph{in})(u, \emph{out})$, edge weight $l(e)=w'(u)$. For the other edges, $l(e)=0$.
\end{itemize}
\item We can see $G'$ is a DAG because splitting vertices cannot create cycles. Let $s'=(s, \emph{in})$. Run the algorithm mentioned in lecture to find the longest path in $G'$ starting from $s'$. Let $d(v)$ be the length of the longest path between $s'$ and $v$. Find the largest $d(v)$ among all $v\in V'$. It is equal to the maximum number of eggs that my friends can collect starting from $s$.
\end{enumerate}
The size of $G'$ is $|V'|=2|V^{\text{SCC}}| and |E'|=|E^{\text{SCC}}|+|V^{\text{SCC}}|$. Since the algorithm for finding the longest path in a DAG runs in $O(m+n)$ time and computing meta-graph also runs in $O(m+n)$ time, this algorithm runs in $O(m+n)$ time where $m=|E|$ and $n=|V|$.

\item We begin by topologically sorting $G^{\text{SCC}}$. We observe that if $uv$ is an edge in the topological sort, then the length of the longest path starting from $u$ must be larger than the one starting from $v$. Therefore, for all vertices reachable from $u$, we only need to calculate once. Following is the algorithm.
\begin{algo}
	\textsc{\textul{MaxEggs($G$):}}\+
\\	compute meta-graph $G^{\text{SCC}}$
\\	do DFS to get a topological sort of $G^{\text{SCC}}$
\\	mark all vertices in $G^{\text{SCC}}$ as unvisited
\\	$\emph{maxEggs}\gets 0$
\\	for each $v\in G^{\text{SCC}}$ in topological order\+
\\	if $v$ is unvisited\+
\\	run the algorithm described in (a) using $v$ as the start vertex
\\	$\emph{maxEggs}\gets \max{(\emph{maxEggs}, \text{computed maximum number of eggs starting at $v$})}$
\\	do a whatever-first search on $G^{\text{SCC}}$, mark all vertices reachable from $v$ as visited
\end{algo}
We can see that although we potentially run the algorithm in (a) multiple times, the parameter of each run is a smaller subgraph of $G^{\text{SCC}}$, and the sum of the sizes of the subgraphs is exactly equal to the size of $G^{\text{SCC}}$. Since the algorithm in (a) runs in $O(m+n)$ time, the algorithm above runs in $O(m+n)+\sum_{i}{O(m_i+n_i)}$ time where $\sum_{i}{m_i}=m$ and $\sum_{i}{n_i}=n$. Therefore, the overall time complexity is still $O(m+n)$.

\item For this problem, we use $G^{\text{SCC}}$ and vertex weights $w'(u)=\sum_{v\in u}{w(v)}$ for vertex $u\in G^{\text{SCC}}$. Let $s$ be an arbitrary start node. Let $\emph{maxEggs}(v,k)$ be the maximum possible number of eggs collected at at most $k$ vertices in the path from $s$ to $v$. Consider the following three cases:
\begin{itemize}
\item We have collected eggs at less than $k$ locations. Then $\emph{maxEggs}(v,k)=\emph{maxEggs}(v,k-1)$.
\item We have collected eggs at exactly $k$ locations, and we collect eggs at $v$. Then $\emph{maxEggs}(v,k)=\max_{uv\in E^{\text{SCC}}}{(\emph{maxEggs}(u, k-1)+w'(v))}$.
\item We have collected eggs at exactly $k$ locations, and we do not collect eggs at $v$. Then $\emph{maxEggs}(v,k)=\max_{uv\in E^{\text{SCC}}}{(\emph{maxEggs}(u, k))}$.
\end{itemize}
The base case is when $k=0$ and we cannot get any eggs. Also, $\emph{maxEggs}(s,k)=w'(s)$ for all $k>0$ because we can only collect eggs at $s$. Following are the recurrences.
\begin{equation*}
\emph{maxEggs}(v, k)=
\begin{cases}
	0 &k=0
\\	w'(s) &v=s, k>0
\\	\max
\begin{cases}
	\emph{maxEggs}(v,k-1)
\\	\max_{uv\in E^{\text{SCC}}}{(\emph{maxEggs}(u, k-1)+w'(v))}
\\	\max_{uv\in E^{\text{SCC}}}{(\emph{maxEggs}(u, k))}
\end{cases}
&\text{otherwise}
\end{cases}
\end{equation*}
The idea is similar to (b): if there is an edge $uv$ in the topological sort, then the maximum number of eggs starting from $u$ must be larger than or equal to the maximum number of eggs starting from $v$, since one can always ignore the eggs in $u$ and start from $v$. Therefore we only need to consider cases where we start at a vertex with no in-edge.
\begin{algo}
	\textsc{\textul{MaxEggsAtMostKLocations($G, k$):}}
\\	compute meta-graph $G^{\text{SCC}}=(V^{\text{SCC}}, E^{\text{SCC}})$
\\	$n\gets |V^{\text{SCC}}|$
\\	integer $\emph{maxEggs}[n][k+1]$
\\	initialize $\emph{maxEggs}[v][0]=0$ for $v\in V^{\text{SCC}}$
\\	for $i\gets 1$ to $k$\+
\\	for $v\in V^{\text{SCC}}$ in topological order\+
\\	if $v$ has no in-edge\+
\\	$\emph{maxEggs}[v][i]\gets w'(v)$\-
\\	else\+
\\	$\emph{maxEggs}[v][i]\gets \emph{maxEggs}[v][i-1]$
\\	for each edge $uv$ in $E^{\text{SCC}}$\+
\\	$\emph{maxEggs}[v][i]\gets \max(\emph{maxEggs}[v][i], \emph{maxEggs}[u][i-1]+w'(v))$
\\	$\emph{maxEggs}[v][i]\gets \max(\emph{maxEggs}[v][i], \emph{maxEggs}[u][i])$\-\-\-\-
\\	return $\max_{v\in V^{\text{SCC}}}{(\emph{maxEggs}[v][k])}$
\end{algo}
To check if a vertex has any in-edge, we can create an $O(n)$ space array initialized with $0$. Then we scan all edges $uv$ and set the $v^{\emph{th}}$ element to $1$ in $O(m)$ time. Then checking for in-edge takes $O(1)$ time for each vertex. Since we scan all vertices and all edges $k$ times with constant-time operations during each scan, the time complexity of this algorithm is $O(k(m+n))$.
\end{enumerate}
\end{solution}




\end{document}
